\section{Введение}

Предположим, мы находимся в следующей общей ситуации анализа данных: была обнаружена случайная выборка $\mathbf{x} = (x_1, x_2, \cdots, x_n)$ из неизвестного распределения вероятностей $F$, и мы хотим оценить интересующий параметр $\theta = t (F)$ на основе $\mathbf{x}$. Для этого мы вычисляем оценку $\hat\theta = s (\mathbf{x})$ из $\mathbf{x}$. [Обратите внимание, что $s (\mathbf{x})$ может быть плагин оценкой $t (\hat F)$, но это не обязательно.] Насколько точна $\hat\theta$? Бутстреп был представлен в 1979 году как компьютерный метод оценки стандартной ошибки $\hat\theta$. Он имеет то преимущество, что он полностью автоматический. Самостоятельная оценка стандартной ошибки не требует теоретических вычислений и доступна независимо от того, насколько математически сложной может быть оценка $\hat\theta = s (\mathbf{x})$. Это описано и проиллюстрировано в этой главе. 