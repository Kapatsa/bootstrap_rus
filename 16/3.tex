\section{Связь перестановочного теста и бутстрепа}

Предыдущий пример иллюстрирует некоторые важные различия между перестановочным тестом и бутстреп проверкой гипотез. Перестановочный тест использует особую симметрию, которая существует при выполнении нулевой гипотезы, для создания перестановочного распределения статистики критерия. Например, в задаче с двумя выборками при проверке гипотезы $F = G$ все перестановки порядковой статистики объединенной выборки равновероятны. В результате этой симметрии $\text{ASL}$ из перестановочного теста является точным: в задаче с двумя выборками $\text{ASL}_{\text{perm}}$ представляет собой точную вероятность получения такой экстремальной статистики теста, как наблюдаемая, с фиксированными значениями данных объединенной выборки.

Напротив, бутстреп явно оценивает вероятностный механизм при нулевой гипотезе, а затем генерирует выборку из него для оценки $\text{ASL}$. Оценка $\text{ASL}_{\text{boot}}$ не интерпретируется как точная вероятность, но, как и все бутстреп оценки, точность гарантирована только тогда, когда размер выборки стремится к бесконечности. С другой стороны, бутстреп проверка гипотез не требует особой симметрии, которая требуется для проверки перестановочным методом, и поэтому может применяться гораздо более широко. Например, в задаче с двумя выборками тестирование гипотезы перестановочным методом может проверять только нулевую гипотезу $F = G$, в то время как бутстреп может проверять равенство средних и дисперсии или равенство средних при возможно неравных дисперсиях. 