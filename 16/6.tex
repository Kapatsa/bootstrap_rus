\section{Обсуждение}

Как показывают примеры в этой главе, при проведении бутстреп проверки гипотез мы должны выбрать две величины:
\begin{enumerate}[(a)]
    \item Тестовую статистику $t(\mathbf{x})$.
    \item Нулевое распределение $\hat{F}_0$ для данных подчиняющихся $H_0$.
\end{enumerate}

Учитывая это, мы генерируем $B$-бутстреп значений $t(\mathbf{x}^{*})$ при $\hat{F}_0$ и оцениваем достигнутый уровень значимости как
\begin{equation}\label{eq16.24}
    \widehat{\text{ASL}}_{\text{boot}} = \#\left\{t(\mathbf{x}^{*b}) \geq t(\mathbf{x})\right\}/B.
\end{equation}
Как показывает пример о толщине марок, иногда выбор $t(\mathbf{x})$ и $\hat{F}_0$ не очевиден. Сложность выбора $\hat{F}_0$ состоит в том, что в большинстве случаев $H_0$ является сложной гипотезой. В примере о толщине марок $H_0$ относится ко всем возможным плотностям с одной модой. Хорошим выбором для $\hat{F}_0$ является распределение, которое подчиняется $H_0$ и наиболее разумно для наших данных; этот выбор делает тест консервативным, то есть с меньшей вероятностью тест ошибочно отвергнет нулевую гипотезу. В примере с марками мы проверили унимодальность путем генерации выборок из унимодального распределения, которое в большинстве случаев является почти бимодальным. Другими словами, мы использовали наименьшее возможное значение для $\hat{h}_1$, и это делает вероятность в \ref{eq16.21} максимально большой.

Выбор тестовой статистики $t(\mathbf{x})$ будет определять мощность теста, то есть вероятность того, что мы отклоним $H_0$, в случае когда она не верна. В примере с марками, если фактическая плотность генеральной совокупности является бимодальной, но плотность ядра Гаусса не аппроксимирует ее точно, тогда тест, основанный на ширине окна $\hat{h}_1$, не будет иметь высокой мощности.

Бутстреп тесты полезны в ситуациях, когда альтернативная гипотеза не уточняется. В случаях, когда существует параметрическая альтернативная гипотеза, могут быть предпочтительны методы правдоподобия или байесовские методы.