\section{Задача с одной выборкой}

В качестве нашего второго примера рассмотрим задачу с одной выборкой, включающую только мышей, подвергшихся лечению. Предположим, что другие исследователи провели эксперименты, аналогичные нашим, но с гораздо большим количеством мышей, и они наблюдали среднюю продолжительность жизни $129.0$ дней для мышей подвергшихся лечению. Мы могли бы захотеть проверить, было ли среднее значение для группы лечения в таблице 2.1 также  равно $129.0$:
\begin{equation}\label{eq16.8}
    H_0: \mu_z = 129.0.
\end{equation}
Можно использовать одновыборочную версию нормального теста. Предполагая генеральную совокупность имеющую нормальное распределение, при нулевой гипотезе
\begin{equation}\label{eq16.9}
    \bar{z} \sim N(129.0,\,\sigma^{2}/n),
\end{equation}
где $\sigma$ --- стандартное отклонение продолжительности лечения. Наблюдая $\bar{z} = 86.9$, $\text{ASL}$ представляет собой вероятность того, что случайная величина $\bar{z}^{*}$, распределенная согласно \ref{eq16.9}, меньше наблюдаемого значения $86.9$
\begin{equation}\label{eq16.10}
    \text{ASL} = \Phi\left(\frac{86.9-129.0}{\sigma/\sqrt{n}}\right),
\end{equation}
где $\Phi$ --- функция стандартного нормального распределения.

Поскольку $\sigma$ неизвестно, подставим оценку
\begin{equation}\label{eq16.11}
    \bar{\sigma} = \left\{\sum\limits_{1}^{n}(z_i-\bar{z})^2/(n-1)\right\}^{1/2} = 66.8,
\end{equation}
в \ref{eq16.10}, получим
\begin{equation}\label{eq16.12}
    \text{ASL} = \Phi\left(\frac{-42.1}{66.8/\sqrt{7}}\right) = 0.05.
\end{equation}
$T$-критерий Стьюдента дает несколько больший $\text{ASL}$
\begin{equation}\label{eq16.13}
    \text{ASL} = \text{Prob}\left\{t_6 < \frac{-42.1}{66.8/\sqrt{7}}\right\} = 0.07.
\end{equation}
Таким образом, есть несущественные(???) доказательства того, что у мышей подвергшихся лечению в нашем исследовании среднее время выживания составляет менее $129.0$ дней. Двусторонние значения $\text{ASL}$ равны $0.10$ и $0.14$ соответственно.

Обратите внимание, что двухвыборочный перестановочный тест не может использоваться для этой задачи. Если бы у нас были доступны все времена для пролеченных мышей (а не только их среднее значение $129.0$), мы могли бы провести двухвыборочный перестановочный тест на эквивалентность двух популяций. Однако у нас нет данных о всех временах, а мы знаем только их среднее значение и хотим проверить $H_0: \mu_z = 129.0$.

Напротив, можно использовать бутстреп. Мы основываем бутстреп проверку гипотезы на распределении тестовой статистики
\begin{equation}\label{eq16.14}
    t(\mathbf{z}) = \frac{\bar{z}-129.0}{\bar{\sigma}/\sqrt{7}}
\end{equation}
при выполнении нулевой гипотезы $\mu_z = 129.0$. Наблюдаемое значение
\begin{equation}\label{eq16.15}
    \frac{86.9-129.0}{66.8/\sqrt{7}} = -1.67.
\end{equation}
Но каково пригодное нулевое распределение? Нам нужно распределение $F$, которое оценивает совокупность продолжительности лечения при $H_0$. Прежде всего отметим, что эмпирическое распределение $\hat{F}$ не является подходящей оценкой для $F$, потому что оно не подчиняется $H_0$. То есть среднее значение $\hat{F}$ не равно нулевому значению $129.0$. Каким-то образом нам нужно получить оценку генеральной совокупности, которая имеет среднее значение $129.0$. Простой способ --- преобразовать эмпирическое распределение $\hat{F}$ так, чтобы оно имело желаемое среднее значение.\footnote{Другой метод обсуждается в задаче 16.5.} Другими словами, мы используем в качестве оценочного нулевого распределения эмпирическое распределение значений
\begin{equation}\label{eq16.16}
    \widetilde{z}_i = z_i - \bar{z} + 129.0 = z_i+42.1
\end{equation}
для $i = \ies{7}$. Мы делаем выборку $\widetilde{z}^{*}_1,\ldots,\widetilde{z}^{*}_7$ с возвращением из $\widetilde{z}_1,\ldots,\widetilde{z}_7$, и для каждой бутстреп выборки вычисляем статистику
\begin{equation}\label{eq16.17}
    t(\mathbf{\widetilde{z}}^{*}) = \frac{\bar{\widetilde{z}}^{*}-129.0}{\bar{\widetilde{\sigma}}^{*}/\sqrt{7}},
\end{equation}
где $\bar{\widetilde{\sigma}}^{*}$ --- стандартное отклонение бутстреп выборки. Всего в $100$ из $1000$ выборок имеют $t(\mathbf{\widetilde{z}}^{*})$ меньше чем $-1.67$, и, следовательно, достигнутый уровень значимости составляет $100/1000 = 0.10$ в отличие от $0.05$ и $0.07$ для нормального и $t$-теста соответственно.

Обратите внимание, что наш выбор нулевого распределения предполагает, что возможные распределения времени лечения, поскольку среднее время меняется, являются просто преобразованными версиями друг друга. Такое семейство распределений называется семейством преобразований(???). Это предположение также присутствует в нормальном и $t$-тестах; но в этих тестах мы предполагаем, что генеральные совокупности имеют нормальное распределение. В любом случае, может быть разумным логарифмировать время выживания перед проведением анализа, потому что зарегистрированное время жизни с большей вероятностью удовлетворит предположению о семействе преобразований или нормальному семейству.

Существует другой, но эквивалентный метод бутстрепа для задачи с одной выборкой. Мы делаем выборку с возвращением из (не подвергшихся преобразованию) значений $z_1, z_2, \ldots, z_7$ и вычисляем статистику
\begin{equation}\label{eq16.18}
    t(\mathbf{z}^{*}) = \frac{\bar{z}^{*}-129.0}{\bar{\sigma}^{*}/\sqrt{7}},
\end{equation}
где $\bar{\sigma}^{*}$ --- стандартное отклонение бутстреп выборки. Эта статистика такая же, как \ref{eq16.17}, поскольку
\begin{equation*}
    \bar{\widetilde{z}}^{*} - 129.0 = (\bar{z}^{*} - \bar{z} + 129.0) - 129.0 = \bar{z}^{*} - \bar{z} 
\end{equation*}
и стандартные отклонения также равны. Это также показывает эквивалентность между тестированием гипотезы с помощью бутстрепа для задачи с одной выборкой и бутстреп доверительным интервалом, описанным в главе 12. Этот интервал основан на процентилях статистики \ref{eq16.18} вычисленной по  бутстреп выборке из $z_1, z_2, \ldots, z_7$, точно так же, как указано выше. Следовательно, бутстреп-$t$ доверительный интервал состоит из тех значений $\mu_0$, которые не отклоняются бутстреп критерием для проверки гипотезы, описанным выше. Эта общая связь между доверительными интервалами и проверками гипотез более подробно описана в разделе 12.3.

