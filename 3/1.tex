\section{Введение}

Проблемы статистического вывода часто включают оценку некоторого свойства распределения вероятностей $F$ на основе случайной выборки, взятой из $F$. Эмпирическая функция распределения, которую мы будем называть $\hat F$, представляет собой простую оценку всего распределения $F$. Оценка какого-то интересующего свойства $F$, например его среднего значения, медианы или корреляции, заключается в использовании соответствующего свойства $\hat F$. Это «принцип плагина». Как мы увидим в главе 6, метод бутстрепа является прямым применением принципа плагина. 