\section{Некоторые сведения о доверительных интервалах}
Перед тем как начать описывать бутстреп, мы рассмотрим доверительные интервалы в целом, и то какие интервалы называются \textit{точными}. Предположим, что оценка $\widehat{\theta}$  распределена нормально с неизвестным математическим ожиданием $\theta$
\begin{gather}\label{12.8}
\widehat{\theta} \sim \mathrm{N}(\theta ,\ \text{se}^2),
\end{gather}
с известной стандартной ошибкой $\text{se}$. (Над знаком <<$\sim $>>  нет точки, потому что мы предполагаем, что (\ref{12.8}) выполняется точно.) Тогда верно (\ref{12.2}): cлучайная величина равная $\frac{\widehat{\theta} - \theta}{\widehat{\text{se}}}$, имеет нормальное распределение,
\begin{gather}\label{12.9}
Z = \frac{\widehat{\theta} - \theta}{\widehat{\text{se}}} \sim \mathrm{N}(0, 1).
\end{gather}
Равенство 
$\text{Prob}\{|Z|  \le z^{1 - \alpha }\} = 1 - 2 \alpha$ алгебраически эквивалентно:
\begin{gather}\label{12.10}
\text{Prob}_{\theta} \left\{ \theta \in [ \widehat{\theta} - z^{(1 - \alpha)} \cdot \text{se}, \ \widehat{\theta} - z^{(\alpha)} \cdot \text{se}]\right\} = 1 - 2 \alpha.
\end{gather}
Обозначение <<$\text{Prob}_{}$>> подчеркивает, что вычисление вероятности (\ref{12.10}) выполняется с истинным средним равным $\theta$, поэтому $\widehat{\theta} \sim \mathrm{N}(\theta ,\ \text{se}^2)$. 

Для удобства мы будем обозначать доверительные интервалы как $\widehat{\theta}_{\text{lo}} =\widehat{\theta} - z^{1 - \alpha } \cdot \text{se}$ и $\widehat{\theta}_{\text{up}} =\widehat{\theta} - z^{\alpha } \cdot \text{se}$ для интервалов из  (\ref{12.10}). В этом случае мы видим, что интервал $[\widehat{\theta} - z^{1 - \alpha } \cdot \text{se}, \ \widehat{\theta} - z^{\alpha } \cdot \text{se}]$ содержит истинное значение $\theta$  с вероятностью $1 - 2 \alpha$. Точнее, вероятность того, что $\theta$ лежит ниже нижнего предела, в точности равна $\alpha$, как и вероятность того, что $\theta$ лежит выше верхнего предела,
\begin{gather}\label{12.11}
\text{Prob}_{\theta}\left\{ \theta < \widehat{\theta}_{\text{lo}} \right\} = \alpha, 
\ \  \text{Prob}_{\theta} \left\{ \theta > \widehat{\theta}_{\text{up}} \right\} = \alpha.
\end{gather}
Тот факт, что (\ref{12.11}) выполняется для любого возможного значения 
$\theta$ и означает что $1 - 2 \alpha$ доверительный интервал $(\widehat{\theta}_{\text{lo}}, \  \widehat{\theta}_{\text{up}})$ точный. Важно помнить, что $\theta$ является константой в утверждении (\ref{12.11}), а величины $\widehat{\theta}_{\text{lo}}$ и $\widehat{\theta}_{\text{up}}$ случайные.

$1 - 2 \alpha$ доверительный интервал $(\widehat{\theta}_{\text{lo}}, \ \widehat{\theta}_{\text{up}})$ со свойством (\ref{12.11}) называется \textit{равноправным}. Это связано с тем фактом, что ошибка покрытия $2 \alpha$ равномерно распределяется между нижней и верхней границами интервала. Доверительные интервалы почти всегда строятся так, чтобы они были равноправными, и в нашем обсуждении мы будем рассматривать только такие интервалы. Отметим также, что из свойства (\ref{12.11}) вытекает свойство (\ref{12.10}), но не наоборот. То есть (\ref{12.11}) требует, чтобы одностороннее покрытие интервала было $\alpha$  с каждой стороны, а не только общее покрытие $1 - 2 \alpha$. Из за этого интервал принимает правильную форму, то есть увеличивает расстояние в обе стороны от $\widehat{\theta}$. Мы будем стремиться к правильному одностороннему охвату при построении приближенных доверительных интервалов.
