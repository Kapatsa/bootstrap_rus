\section{Бутстреп-t интервал}
Используя бутстреп, мы можем получить точные интервалы, не делая обычных теоретических предположений, таких как (\ref{12.17}). В этом разделе мы описываем способ получить такие интервалы, а именно «бутстреп-t» подход. Эта процедура оценивает распределение $Z$ непосредственно по данным; по сути, он строит таблицу, подобную Таблице (12.1), \textit{которая подходит для имеющегося набора данных}\footnote{Легче описать идею лежащую в основе бутстреп-t метода, нежели основанные на процентилях бутстреп интервалы, которые будут описаны в следующих двух главах, поэтому сначала мы обсудим бутстреп-t. На практике, однако, бутстреп-t может давать несколько ошибочные результаты, и на него могут сильно повлиять несколько отдаленных точек данных. Методы, основанные на процентилях, из следующих глав более надежны}. Затем эта таблица используется для построения доверительного интервала точно так же, как используются таблицы нормального и $t$ - распределения для (\ref{12.17}) и (\ref{12.18}). Бутстреп таблица строится путем создания $B$ бутстреп выборок и последующего вычисления бутстреп версии $Z$ для каждого из них. Бутстреп таблица состоит из процентилей этих значений $B$.

Приведём бутстреп-t метод более подробно. Используя обозначения с рисунка 8.3 мы генерируем $B$ бутстреп выборку $\textbf{x}^{*1},\textbf{x}^{*2},\ldots, \textbf{x}^{*B}$ и для каждой вычисляем:
\begin{gather}\label{12.20}
Z^{*}(b) = \frac{\widehat{\theta}^{*}(b) - \widehat{\theta}}{\widehat{\text{se}}^{*}(b)},
\end{gather}
где $\widehat{\theta}^{*}(b) = s(\textbf{x}^{*b})$ значение $\widehat{\theta}$ для бутстреп выборки $\textbf{x}^{*b}$, а $\widehat{\text{se}}^{*}(b)$ это стандартная ошибка оценки $\widehat{\theta}^{*}(b)$ для бутстреп выборки $\textbf{x}^{*b}$. $\alpha$ процентиль $Z^{*}(b)$, который оценивается как:
\begin{gather}\label{12.21}
\# \left\{Z^{*}(b)\le \widehat{t}^{(\alpha)} \right\} / B = \alpha.
\end{gather}
Например, для $B = 1000$, $5 \%$ процентиль является 50-м самым большим значением $Z^{*}(b)$, а оценка точки $95 \%$ это 950 по величине значение $Z^{*}(b)$. Наконец, доверительный интервал для бутстреп-t метод равен:
\begin{gather}\label{12.22}
( \hat{\theta} - \widehat{t}^{(1 - \alpha)} \cdot \widehat{\text{se}}, \ \widehat{\theta} - \widehat{t}^{(\alpha)} \cdot \widehat{\text{se}}).
\end{gather}
Это вытекает из той же логики, которая используется для получения (\ref{12.19}) из (\ref{12.18}).

Если $B \cdot \alpha $  не целое число, то можно использовать следующую процедуру. Предположим, что $\alpha < 0.5$, пусть $k = [(B + 1) \alpha]$, наибольшее целое число $\le(B + 1) \alpha$.
Затем мы определяем эмпирические квантили $\alpha $ и $1 - \alpha$ используя
$k$ и $(B + 1 - k)$ наибольшее значение $Z^{*}(b)$, соответственно. 
 
Последняя строке таблицы 12.1 отображает процентили $Z^{*}(b)$ для $\widehat{\theta}$, которая является оценкой среднего значения контрольной группы данных для мышей, и вычеслена на основе бутстреп выборки размером $1000$. Важно отметить, что $B = 100$ или $200$ не подходит для построения доверительного интервала, см. Главу 19. Обратите внимание, что бутстреп-t значения сильно отличаются от нормальных и t процентилей! Полученный $90 \%$ bootstrap-t доверительный интервал для среднего будет равен:
$$
[56.22- 1.53 \cdot 13.33, \ 56.22 + 4.53 \cdot 13.33] = [35.82, \ 116.74].
$$
Точка нижней границы близка к стандартному интервалу, но верхняя граничная точка намного больше. Это связано с влиянием двух очень больших значений данных 104 и 146.

Величина $Z = \frac{\widehat{\theta} - {\theta}}{\widehat{\text{se}}}$, называется \textit{основой приближения???}: это означает, что её распределение примерно одинаково для каждого значения $\theta$. Именно это свойство позволяет нам построить интервал (\ref{12.22}) из бутстреповского распределения $Z^{*}(b)$, используя тот же аргумент, который позволил получить (\ref{12.5}) из (\ref{12.3}).

Некоторая теория (глава 22) показывает, что в больших выборках покрытие бутстреп-t интервала имеет тенденцию быть ближе к желаемому уровню (здесь $90 \%$), чем покрытие стандартного или основанного на t - таблице интервала. . Интересно, что, как и в случае t-приближения, повышение точности достигается за счет универсальности. Стандартная нормальная таблица применяется ко всей выборке и выборкам всех размеров; t - таблица применяет ко всем выборкам фиксированного размера $n$; таблица бутстреп-t применяется только \textit{к данной выборке}. Однако при наличии быстрых компьютеров не так сложно вывести <<бутстреп таблицу>> для каждой новой проблемы, с которой мы сталкиваемся.

Обратите внимание, что значения для нормального и t распределения в таблице 12.1 симметричны относительно нуля, и, как следствие, результирующие интервалы симметричны относительно точечной оценки $\widehat{\theta}$. В то время как бутстреп-t процентили  могут быть асимметричными относительно 0, что приводит к удлинению границ интервала. Эта асимметрия и является улучшением данного покрытия.

Бутстреп-t процедура является полезным и интересным обобщением обычного t - метода Стьюдента. Это особенно применимо к \textit{центральным статистикам}, таким как выборочное среднее. Центральная статистика - это статистика, для которой увеличение каждого значения данных $x_{i}$ на константу $c$ влечёт за собой увеличиние самой статистики на $c$. Другой пример центральных статистик - это медиана, усеченное среднее или выборочный процентиль.

Нельзя доверять решения более общих задач, бутстреп-t методу, по крайней мере в его простой форме, например, таких задач как нахождение доверительного интервала для коэффициента корреляции. В следующих двух главах мы расскажем про более надежные методы бутстреп доверительного интервала. В следующем разделе мы опишем использование преобразований для улучшения бутстреп-t подхода.