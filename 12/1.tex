\section{Введение}

До сих пор большая часть нашей работы касалась вычисления бутстреп стандартных ошибок. Стандартные ошибки часто используются для нахождения доверительных интервалов интересующих нас параметров. Учитывая оценку $\widehat{\theta}$ и оценку стандартной ошибки $\widehat{\text{se}}$, $ 90\% $ доверительный интервал для $\theta$:

\begin{gather}\label{12.1}
\widehat{\theta} \pm 1.645 \cdot \widehat{\text{se}}.
\end{gather}
Значение 1.645 взято из стандартной нормальной таблицы, о чем будет кратко сказано ниже. Выражение (\ref{12.1}) называется интервальной оценкой или доверительным интервалом $\theta$. Интервальная оценка часто бывает более полезной, чем просто точечная оценка $\widehat{\theta}$. Рассмотренные вместе, точечная и интервальная оценки говорят о том, какое предположение является наилучшим для $\theta$, и насколько ошибочным может быть это предположение. 

В этой и в двух следующих главах мы описываем различные методы построения доверительных интервалов с использованием бутстрепа. Эта область была основным направлением теоретческой работы по бутстрепу; обзор этой работы приведен далее в книге (глава 22).

Предположим, что мы находимся в ситуации когда у нас одна выборка, полученная случайным отбором из неизвестного распределения $F, \ F \rightarrow \mathbf{x} = (x_{1}, x_{2},..., x_{n})$, как в главе 6. Пусть $\widehat{\theta} = t(\widehat{F})$ дополнительная оценка интересующего параметра $\theta = t(F)$, а $\widehat{\text{se}}$ некоторая оценка стандартной ошибки $\widehat{\theta}$, основанная например на бутстрепе или методе складного ножа. В большинстве случаев оказывается, что по мере увеличения размера выборки $n$ распределение $\widehat{\theta}$ становится всё более похожим на нормальное, со средним значением около $\theta$ и дисперсией около $\widehat{\text{se}}^{2}$, $\widehat{\theta}  \ \dot{\sim}  \ \mathrm{N} \  (\theta, \  \widehat{\text{se}}^{2})$ или равно

\begin{gather}\label{12.2}
\frac{\widehat{\theta} - \theta}{\widehat{\text{se}}} \  \dot{\sim} \ \mathrm{N}(0,1).
\end{gather}
По мере увеличения объёма данных асимптотический результат (\ref{12.2}) становится верным для вероятностных моделей $P \rightarrow \mathbf{x}$ и для статистик, не основанных на выборке из распределения, но мы остановимся на одновыборочном приближении для большинства случаев в этой главе. 

Пусть $z^{\alpha}$ значение $100\alpha$ процентиля распределения $\mathrm{N}(0,1)$, из таблицы стандартного нормального распределения, $z^{0.025} = -1.960,$  $z^{0.05} = -1.645,$ $z^{0.95} = 1.645,$ $z^{0.975} = 1.960$, и т.д.
Если считать приближение (\ref{12.2}) точным, то
\begin{gather}\label{12.3}
\text{Prob}_{F} \left \{ z^{\alpha} \le \frac{\widehat{\theta} - \theta}{\widehat{\text{se}}} \le z^{1 - \alpha }\right\} = 1 - 2 \alpha,
\end{gather}
что может быть переписано как:
\begin{gather}\label{12.4}
\text{Prob}_{F}\left\{ \theta \in [ \widehat{\theta} - z^{1 - \alpha } \cdot \widehat{\text{se}}, \ \widehat{\theta} - z^{\alpha } \cdot \widehat{\text{se}} ] \right\} = 1 - 2 \alpha.
\end{gather}
Интервал (\ref{12.1}) получается из (\ref{12.4}), c $\alpha = 0.05,$ $1 - 2 \alpha = 0.90.$ Выражение
\begin{gather}\label{12.5}
[ \widehat{\theta} - z^{1 - \alpha } \cdot \widehat{\text{se}},\ \widehat{\theta} - z^{\alpha} \cdot \widehat{\text{se}}]
\end{gather}
\textit{называется стандартным доверительным интервалом} c \textit{вероятностью охвата}\footnote{Точнее было бы называть \ref{12.5} \underline{приблизительным} доверительным интервалом, так как вероятность охвата скорее всего не будет точно равна значению $100 \cdot(1 - 2 \alpha)$. Интервалы бутстреп, обсуждаемые в этой главе, также являются приблизительными, но в целом они лучше, чем стандартные интервалы.} равной $1 - 2 \alpha$, или \textit{уровнем доверия} $100 \cdot(1 - 2 \alpha) \%$. Или это называется $1 - 2 \alpha$ доверительным интервалом для $\theta$. Поскольку $ z^{\alpha } = -z^{1 - \alpha}$, мы можем написать (\ref{12.5}) в более знакомом виде
\begin{gather}\label{12.6}
\widehat{\theta} \pm z^{1 - \alpha} \cdot \widehat{\text{se}}.
\end{gather}
В качестве примера рассмотрим $n = 9$ мышей контрольной группы из таблицы 2.1. Предположим, что нам нужен доверительный интервал математического ожидания $\theta$ для распределения контрольной группы. Оценка методом подстановки $\widehat{\theta} = 56.44$ c оценкой стандартного отклонения $\widehat{\text{se}} = 13.33 $ как в (5.12). $90 \%$ доверительный интервал $\theta$ (\ref{12.1}) это
\begin{gather}\label{12.7}
56.22 \pm 1.645\cdot 13.33 = [34.29,\ 78.15].
\end{gather}

Покрывающее свойство этого интревала означает что в $90\% $ случаев случайный интервал, построенный таким образом, будет содержать истинное значение $\theta.$ Конечно, в большинстве задач (\ref{12.2}) является только приближением, а стандартный интервал --- это только приближение доверительного интервала, хотя часто он может быть полезным. Мы используем бутстреп для расчета более точных оценок доверительных интервалов. Поскольку $n \rightarrow \infty$, границы бутстреп и стандартного интервала сходятся друг к другу, но в ситуации когда это не так, как например в ситуации с данными о мышах, бутстреп может внести значительные поправки. Эти поправки могут значительно повысить точность оценки доверительного интервала. 