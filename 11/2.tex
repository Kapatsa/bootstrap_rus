\section{Определение складного ножа}

Предположим, у нас есть выборка $x = (x_1, x_2, \dots, x_n)$ и оценка $\hat{\theta} = s(x)$. Мы хотим оценить смещение и стандартную ошибку $\hat{\theta}$. Складной нож фокусируется на выборках, которые не учитывают одно наблюдение за раз:
\begin{equation}\label{eq11.1}
    x_{(i)} = (x_1, x_2, \dots, x_{i-1}, x_{i+1}, \dots, x_n)
\end{equation}
для $i = 1, 2, \dots, n$, так называемых выборках складного ножа. Выборка складного ножа под номером $i$ состоит из набора данных с удаленным $i$-м наблюдением. Пусть
\begin{equation}\label{eq11.2}
    \hat{\theta}_{(i)} = s(x_{(i)})
\end{equation}
будет $i$-й репликацией складного ножа $\hat{\theta}$.

Оценка смещения по методу складного ножа определяется как
\begin{equation}\label{eq11.3}
   \widehat{\text{bias}}_{\text{jack}} = (n-1)\left(\hat{\theta}_{(\cdot)} - \hat{\theta}\right),
\end{equation}
где
\begin{equation}\label{eq11.4}
   \hat{\theta}_{(\cdot)} = \sum\limits_{i=1}^{n}\hat{\theta}_{(i)}/n.
\end{equation}
Оценка стандартной ошибки по методу складного ножа определяется как
\begin{equation}\label{eq11.5}
   \widehat{\text{se}}_{\text{jack}} = \left[\frac{n-1}{n}\sum\left(\hat{\theta}_{(i)} - \hat{\theta}_{(\cdot)}\right)^{2}\right]^{1/2}.
\end{equation}

Откуда берутся эти формулы? Начнем с $\widehat{\text{se}}_{\text{jack}}$. Вместо того, чтобы смотреть на все (или некоторые) наборы данных, которые могут быть получены путем выборки с заменой из $x_1, x_2, \dots, x_n$, складной нож рассматривает $n$ фиксированных выборок $x_{(1)}, \dots, x_{(n)}$, полученные удалением по одному наблюдению за раз. Подобно бутстреп оценке стандартной ошибки, формула для $\widehat{\text{se}}_{\text{jack}}$ выглядит как стандартное отклонение выборки этих $n$ значений, за исключением того, что первый коэффициент равен $(n-1)/n$ вместо $1/(n-1)$ или $1/n$. Конечно, $(n-1)/n$ намного больше, чем $1/(n-1)$ или $1/n$. Интуитивно этот <<коэффициент увеличения>> необходим, потому что отклонения складного ножа
\begin{equation}\label{eq11.6}
   \left(\hat{\theta}_{(i)} - \hat{\theta}_{(\cdot)}\right)^{2}
\end{equation}
имеют тенденцию быть меньше, чем бутстреп отклонения
\begin{equation}\label{eq11.7}
   \left[\hat{\theta}^{*}(b) - \hat{\theta}^{*}(\cdot)\right]^{2},
\end{equation}
поскольку типичная выборка метода складного ножа больше похожа на исходные данные $x$, чем типичная бутстреп выборка.

Точный вид множителя $(n-1)/n$ получается путем рассмотрения частного случая $\hat{\theta} = \Bar{x}$. Тогда легко показать, что
\begin{equation}\label{eq11.8}
   \widehat{\text{se}}_{\text{jack}} = \left\{\sum\limits_{1}^{n}(x_i-\bar{x})^2/\{(n-1)n\}\right\}^{1/2}.
\end{equation}
То есть коэффициент $(n-1)/n$ --- это именно то, что нужно, чтобы сделать $\widehat{\text{se}}_{\text{jack}}$ равным несмещенной оценке стандартной ошибки среднего. Коэффициент $[(n-1)/n]^2$ приведет оценке метода подстановки
\begin{equation}\label{eq11.9}
   \left\{\sum\limits_{1}^{n}(x_i-\bar{x})^2/n^{2}\right\}^{1/2},
\end{equation}
но это существенно не отличается от несмещенной оценки, если только $n$ не мало. Соглашение о том, что $\widehat{\text{se}}_{\text{jack}}$ использует множитель $(n-1)/n$,  несколько произвольно.

Аналогичным образом, оценка смещения по методу складного ножа (\ref{eq11.3}) кратна среднему значению отклонений складного ножа
\begin{equation}\label{eq11.10}
   \hat{\theta}_{(i)} - \hat{\theta}, \quad i = 1, 2, \dots, n.
\end{equation}
Величины (\ref{eq11.10}) иногда называют величинами влияния складного ножа. Обратите внимание на множитель $(n-1)$ в (\ref{eq11.3}). Это коэффициент увеличения, аналогичный тому, который появляется при оценке по методу складного ножа стандартной ошибки. Чтобы вывести его, мы не можем обратиться к частному случаю $\hat{\theta} = \bar{x}$, потому что $\bar{x}$ несмещенная, а $\hat{\theta}_{(\cdot)} - \hat{\theta}$, как и должно быть, равно нулю. Поскольку этот случай не говорит нам, каким должен быть старший фактор, мы вместо этого рассматриваем в качестве нашего тестового примера выборочную дисперсию
\begin{equation}\label{eq11.11}
   \hat{\theta} = \sum\limits_{1}^{n}(x_{i}-\bar{x})^{2}/n.
\end{equation}
Она имеет смещение $-1/n$ дисперсий генеральной совокупности, а множитель $(n-1)$ перед $\hat{\theta}_{(\cdot)} - \hat{\theta}$ делает $\widehat{\text{bias}}_{\text{jack}}$ равным $-1/n$, умноженному на $$\sum(x_i-\bar{x})^2/(n-1),$$ несмещенной оценке дисперсии генеральной совокупности.
