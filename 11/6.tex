\section{Отказ складного ножа}

Подводя итог, можно сказать, что складной нож часто обеспечивает простое и хорошее приближение к бутстрепу для оценки стандартных ошибок и смещения. Однако, как вкратце упоминалось в главе 10, складной нож может с треском выйти из строя, если статистика $\hat{\theta}$ не является «гладкой». Интуитивно идея гладкости заключается в том, что небольшие изменения в наборе данных вызывают только небольшие изменения в статистике. Простым примером негладкой статистики является медиана. Чтобы понять, почему медиана не является гладкой, рассмотрим $9$ упорядоченных значений из контрольной группы данных о мышах (таблица 2.1):
\begin{equation}\label{eq11.19}
    10,27,31,40,46,50,52,104,146.
\end{equation}
Медиана этих значений равна $46$. Теперь предположим, что мы начинаем увеличивать значение $4$-го по величине значения $x = 40$. Медиана не меняется вообще, пока $x$ не станет больше $46$, а затем после этого медиана будет равна $x$ , пока $x$ не превысит $50$. Это означает, что медиана не является дифференцируемой (или гладкой) функцией от $x$.

Это отсутствие гладкости приводит к тому, что оценка стандартной ошибки по методу складного ножа несовместима с медианой. Для данных о мышах значения складного ножа для медианы\footnote{Медиана четного числа точек данных --- это среднее двух значений из середины.} равны
\begin{equation}\label{eq11.20}
    48,48,48,48,45,43,43,43,43.
\end{equation}
Обратите внимание, что встречаются только $3$ различных значения, что является следствием недостаточной гладкости медианы и того факта, что наборы данных складного ножа отличаются от исходного набора данных только на одно наблюдение. Итоговая оценка $\mathrm{se}_{\mathrm{jack}}$ составляет $6.68$. Для данных о мышах бутстреп оценка стандартной ошибки на основе бутстреп выборок объема $B = 100$ составляет $9.58$, что значительно больше, чем значение складного ножа, равное $6.68$. При $n \rightarrow \infty$, можно показать, что $\mathrm{se}_{\mathrm{jack}}$ противоречива, то есть не может сходиться к истинной стандартной ошибке. С другой стороны, бутстреп рассматривает наборы данных, которые менее похожи на исходный набор данных, чем наборы данных складного ножа, и, следовательно, согласованы с медианой. 