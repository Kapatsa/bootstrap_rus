\section{Бутстреп оценка смещения}

Для начала предположим, что мы вернулись к ситуации с непараметрической выборкой, как в главе 6. Неизвестное распределение вероятностей $F$ дает набор $x = (x_1, x_2 \dots, x_n)$ путем случайной выборки $F \rightarrow x$. Мы хотим оценить вещественный параметр $\theta = t(F)$. Пока возьмем за оценку любую статистику $\hat{\theta} = s(x)$, как показано на рисунке 6.1. Позже нас будет особенно интересовать оценка плагина $\hat{\theta} = t(\hat{F})$.

Смещение $\hat{\theta} = s(x)$ как оценка $\theta$ определяется как разница между математическим ожиданием $\hat{\theta}$ и значением параметра $\theta$,
\begin{equation}\label{eq10.1} 
    \text{bias}_{F} = \text{bias}_{F}(\hat{\theta}, \theta) = E_{F}[s(x)] - t(F).
\end{equation}

Большое смещение обычно является нежелательным аспектом поведения оценки. Мы смирились с тем фактом, что $\hat{\theta}$ является непостоянной оценкой $\theta$, но обычно мы не хотим, чтобы изменчивость была исключительно низкой или высокой. Несмещенные оценки те, для которых $E_{F}(\hat{\theta}) = \theta$, играют важную роль в статистической теории и практике. Они способствуют хорошему чувству научной объективности в процессе оценки. Оценки плагина $\hat{\theta} = t(\hat{F})$ необязательно являются несмещенными, но они, как правило, имеют небольшие смещения по сравнению с величиной их стандартных ошибок. Это одна из хороших черт принципа плагина.

Мы можем использовать бутстреп чтобы вычислить смещение любой оценки $\hat{\theta} = s(x)$. Бутстреп оценка смещения определяется как оценка смещения, которую мы получаем, подставляя $\hat{F}$ вместо $F$ в \ref{eq10.1},
\begin{equation}\label{eq10.2} 
    \text{bias}_{\hat{F}} = E_{\hat{F}}[s(x^{*})] - t(\hat{F}).
\end{equation}

Здесь $t(\hat{F})$ --- оценка плагина $\theta$ может отличаться от $\hat{\theta} = s(x)$. Другими словами, $\text{bias}_{\hat{F}}$ --- это оценка плагина $\text{bias}_{F}$ независимо от того, является ли $\hat{\theta}$ оценкой плагина $\theta$ или нет. Обратите внимание, что $\hat{F}$ используется дважды при переходе от \ref{eq10.1} к \ref{eq10.2}: при замене $F$ в $t(F)$ и при замене $F$ в $E_{F}[s(x)]$.

Если $s(x)$ --- среднее значение, а $t(F)$ --- среднее значение генеральной совокупности, легко показать, что $\text{bias}_{\hat{F}} = 0$. Это имеет смысл, потому что среднее --- это несмещенная оценка среднего для генеральной совокупности, то есть $\text{bias}_{F} = 0$. Однако обычно статистика имеет некоторую систематическую ошибку, и $\text{bias}_{\hat{F}}$ дает оценку этой систематической ошибки. Простым примером является выборочная дисперсия $s(x) = \sum_{1}^n(x_{i} - \Bar{x})^{2}/n$, погрешность которой равна $(-1/n)$ дисперсии генеральной совокупности. В этом случае легко показать, что $\text{bias}_{\hat{F}} = (-1/n^{2})\sum_{1}^n(x_{i} - \Bar{x})^{2}$.

Для большинства статистик, которые возникают на практике, идеальная бутстреп оценка $\text{bias}_{\hat{F}}$ должна быть аппроксимирована моделированием Монте-Карло. Мы генерируем независимую бутстреп выборку $x^{*1}, x^{*2}, \dots x^{*B}$ как на рисунке $6.1$, вычисляем бутстреп репликации $\hat{\theta}^{*}(b) = s(x^{*b})$ и аппроксимируем бутстреп математическое ожидание $E_{\hat{F}}[s(x^{*})]$ средним
\begin{equation}\label{eq10.3} 
    \hat{\theta^{*}}(\cdot) = \sum\limits_{b=1}^{B}\hat{\theta^{*}}(b)/B = \sum\limits_{b=1}^{B} s(x^{*b})/B.
\end{equation}
Бутстреп оценка смещения, основанная на $B$ репликах $\widehat{\text{bias}}_{B}$, есть \ref{eq10.2} с заменой $E_{\hat{F}}[s(x^{*})]$ на $\hat{\theta^{*}}(\cdot)$,
\begin{equation}\label{eq10.4} 
   \widehat{\text{bias}}_{B} = \hat{\theta^{*}}(\cdot) - t(\hat{F}).
\end{equation}
Обратите внимание, что алгоритм, показанный на рисунке $6.1$, точно применяется к вычислению \ref{eq10.4}, за исключением того, что на последнем шаге мы вычисляем $\hat{\theta^{*}}(\cdot) - t(\hat{F})$, а не $\widehat{se}_{B}$. Конечно, мы можем вычислить как $\widehat{se}_{B}$, так и $\widehat{\text{bias}}_{B}$ с помощью того же набора бутстреп репликаций.