\section{Введение}

Мы сосредоточились на стандартной ошибке как на показателе точности оценки $\hat{\theta}$. Существуют и другие пригодные меры статистической точности (или статистической ошибки), измеряющие различные аспекты поведения оценок  $\hat{\theta}$. В этой главе речь идет о смещении, разнице между ожидаемым значением оценки $\hat{\theta}$ и оцениваемой величиной $\theta$. Алгоритм бутсрепа легко адаптируется для получения оценок смещения, ровно как и для получения оценок стандартной ошибки. Также вводится оценка смещения по методу складного ножа, хотя мы откладываем полное обсуждение метода складного ножа до главы 11. Можно использовать оценку смещения для исправленной оценки смещения. Однако это может быть опасной практикой, о чем говорится в конце главы.
