\section{Введение}

Перестановочные методы --- это трудоемкий компьютерный статистический метод, появившийся еще до возникновения компьютеров. Идея была предложена Р.А. Фишером в 1930-х годах, скорее как теоретический аргумент в пользу t-критерия Стьюдента, нежели чем самостоятельный полезный статистический метод. Современные вычислительные мощности делают перестановочные тесты практичными для повседневного использования. Основная идея привлекательно проста и свободна от математических предположений. Существует тесная связь с бутстрепом, который обсуждается далее в этой главе.