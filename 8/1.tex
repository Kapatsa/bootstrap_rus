\section{Введение}

Алгоритм бутстрепа на рисунке 6.1 основан на простейшей возможной вероятностной модели для случайных данных: одновыборочная модель, в которой одно неизвестное вероятностное распределение $F$ порождает данные $\textbf{x}$ путем случайной выборки
\begin{equation}
	F \to \textbf{x} = (x_1, x_2, \ldots , x_n).
\end{equation}
Отдельные элементы $x_i$ в (8.1) сами по себе могут быть довольно сложными, возможно, в виде чисел, векторов, карт, изображений или чего-то еще, но сам вероятностный механизм прост. Многие задачи анализа данных связаны с более сложными структурами данных. Эти такие структуры как временные ряды, дисперсионный анализ, регрессионные модели, многовыборочные задачи, цензурированные данные, стратифицированная выборка и т.д. Алгоритм бутстрепа можно адаптировать к общим структурам данных, как это обсуждается здесь и в главе 9.