\section{Метод $\bca$}

В этом разделе описано построение $\bca$ интервалов. Они оказываются более сложными в описании, чем процентильные интервалы, однако в применении они так же просты. Алгоритм \texttt{bcanon}, данный в приложении, строит непараметрические $\bca$ интервалы адаптивно.

Пусть $\what \theta^{*(\alpha)}$ обозначает $100\cdot \alpha$ процентиль $B$ бутстреп репликаций $$\what\theta^*(1),\what\theta^*(2),\ldots,\what\theta^*(B),$$ как в (13.5). Процентильный интервал $(\what \theta_\tx{lo}, \what \theta_\tx{up})$ предполагаемого покрытия $1-2\alpha$, получается напрямую из этих процентилей, то есть
$$
\tx{процентильный метод: } \quad (\what \theta_\tx{lo}, \what \theta_\tx{up}) = 
(\what \theta^{*(\alpha)}, \what \theta^{*(1 - \alpha)}).
$$
Например, пусть $B = 2000$ и $\alpha = 0.05$; тогда процентильный интервал $(\what \theta^{*(0.05)}, \what \theta^{*(0.95)})$ будет интервалом, покрывающим упорядоченные значения $\what \theta^*(b)$ от 100го до 1900го.

Границы интервала $\bca$ также даются процентилями бутстреп распределения, однако они необязательно совпадают с интервалом на основе (14.8). Используемые процентили зависят от двух чисел $\hat a$ и $\hat z_0$, которые определены как \textit{ускорение} (acceleration) и \textit{поправка смещения} (bias-correction), соответственно. Далее мы опишем получение чисел $\hat a$ и $\hat z_0$, но сначала дадим определение границ интервала $\bca$.

\textit{$\bca$ интервал с предполагаемым покрытием $1-2\alpha$ задается парой значений}
\begin{equation}
  	\bca\, : \, (\what \theta_\tx{lo}, \what \theta_\tx{up}) = 
(\what \theta^{*(\alpha_1)}, \what \theta^{*(\alpha_2)}),
\end{equation}
\textit{где}
\begin{align}
	\alpha_1 &= \Phi \left(\hat z_0 + \frac{\hat z_0 + z^{(\alpha)}}{1 - \hat a (\hat z_0 + z^{(\alpha)})}\right), \notag \\
	 \alpha_2 &= \Phi \left(\hat z_0 + \frac{\hat z_0 + z^{(1 -\alpha)}}{1 - \hat a (\hat z_0 + z^{(1 -\alpha)})}\right).
\end{align}
Здесь $\Phi(\cdot)$ есть функция стандартного нормального распределения, a $z^{(\alpha)}$ есть $100\alpha$ процентиль стандартного нормального распределения. К примеру, $z^{(0.95)} = 1.645$ и $\Phi(1.645) = 0.95$.

Формула (14.10) выглядит сложно, однако её легко вычислить. Заметим, что если приравнять $\hat a$ и $\hat z_0$ к нулю, то 
\begin{equation}
  \alpha_1 = \Phi(z^{\alpha}) = \alpha \quad \tx{и} \quad \alpha_1 = \Phi(z^{1-\alpha}) = 1-\alpha,
\end{equation}
откуда можно увидеть, что в таком случае $\bca$ интервал (14.9) совпадает с процентильным интервалом (13.4). Ненулевые значения $\hat a$ или $\hat z_0$ изменяют процентили, используемые для вычисления границ $\bca$. Такие изменения исправляют некоторые недостатки стандартного и процентильного методов, что объясняется в 22 главе.
Непараметрические $\bca$ интервалы из таблицы 14.2 построены на значениях
\begin{equation}
  (\hat a, \hat z_0) = (0.061, 0.146),
\end{equation}
что приводит к значениям (согласно (14.10))
\begin{equation}
  (\alpha_1, \alpha_2) = (0.110,0.985).
\end{equation}
В данном случае $90\%$ $\bca$ интервал есть $
(\what \theta^{*(0.110)}, \what \theta^{*(0.985)})$, интервал, расположенный между 220-ым и 1970-ым  упорядоченными значениями 2000 чисел $\what \theta^* (b)$.

Как вычисляются $\hat a$ и $\hat z_0$? Значение поправки смещения $\hat z_0$ получается напрямую из доли бутстреп репликаций, меньших исходной оценки $\what \theta$,
\begin{equation}
  \hat z_0 = \Phi^{-1} \left(\frac{\#\left\{\what\theta^*(b) < \what \theta\right\}}{B}\right),
\end{equation}
где $\Phi^{-1}(\cdot)$ есть обратная функция к функции распределения стандартного нормального закона.\footnote{то есть $\Phi^{-1}(0.95) = 1.645$} У левой гистограммы на рисунке 14.2 1116 из 2000 значений $\what \theta^*$ оказались меньше, чем $\what \theta = 171.5$, откуда $\hat z_0 = \Phi^{-1}(0.558) = 0.146$. Грубо говоря, $\hat z_0$ оценивает медианное смещение $\what \theta^*$, то есть степень различия между медианой $\what \theta^*(b)$ и $\what \theta$ в <<нормальной>> шкале. Мы получим $\hat z_0$, если ровно половина из всех значений $\what \theta^*(b)$  окажется меньшими или равными $\what \theta$.

Есть несколько способов вычисления ускорения $\hat a$. Проще всего описать его с помощью значений по методу складного ножа статистики $\what \theta = s(\mbf x)$. Пусть $\mbf x_{(i)}$ --- исходная выборка с удаленным наблюдением $x_i$, также обозначим $\what \theta _{(i)} := s(\mbf x_{(i)})$  и $\what \theta_{(\cdot)} = \summ{i = 1}{n} \what \theta_{(i)}/n$, согласно рассуждениям из начала главы 11. Простое выражение для ускорения
\begin{equation}
  \hat a = \frac{\summ{i = 1}{n}\left(\what \theta_{(\cdot)} - \what \theta_{(i)}\right)^3}{6 \left\{\summ{i = 1}{n}\left(\what \theta_{(\cdot)} - \what \theta_{(i)}\right)^2\right\}^{3/2}}.
\end{equation}
У статистики $s(\mbf x) = \summ{i=1}{n} (A_i - \bar A)^2/n$, из (14.2), значение $\hat a$ для набора данных о тестах на пространственное восприятие соатавляет $\hat a = 0.061$. Как $\hat a$, так и $\hat z_0$ вычисляются автоматически реализацией непараметрического алгоритма $\bca$. Величина $\hat a$ называется \textit{ускорением} из-за того, что она описывает скорость изменения стандартной ошибки $\what \theta$ относительно истинного значения параметра $\theta$. Стандартная нормальная аппроксимация --- $\what \theta \sim N(\theta, \tx{se}^2)$ --- предполагает, что стандартная ошибка $\what \theta$ одинакова для всех $\theta$. Однако часто это предположение нереалистично, и константа ускорения $\hat a $ делает поправку. Например, в текущем примере, где $\what \theta$ есть дисперсия, в контексте теории нормального распределения ясно, что $\tx{se} \what \theta \sim \theta$ %(Задача 14.4)
Фактически, $\hat a $ есть скорость изменения стандартной ошибки $\hat \theta$ относительно истинного значения параметра $\theta$, измеренная в <<нормальной>> шкале. Не является очевидным то, почему формула (14.15) должна привести к оценке ускорения стандартной ошибки: некоторые разъяснения этого результата можно найти у Efron (1987).

У метода $\bca$ есть два важных теоретических преимущества. Во-первых, этот метод сохраняет отображения,\footnote{Данное утверждение будет строго верным, если принять другое определение $\hat a$, основанное на конечных разностях, как в главе 22. На практике это различие оказывается несущественным} как в формуле (13.10). Это означает, что граничные точки интервала $\bca$ отображаются корректно при замене интересующего параметра $\theta$ на некоторую функцию от него. Например, $\bca$ интервалы для $\sqrt{\tx{var}(A)}=\sqrt{\theta}$ получаются взятием квадратных корней из граничных точек $\bca$ в таблице 14.2. Свойство сохранения интервала при отображении оберегает от сомнений, которые имеют место при выборе масштаба для бутстреп-t интервала, как в разделе 12.6. $\bca$ автоматически выбирает наилучшую шкалу.

Второе преимущество метода $\bca$ заключается в его точности. Доверительный интервал $(\what \theta_\tx{lo}, \what \theta_\tx{up})$ уровня $1 - 2\alpha$  должен иметь вероятность $\alpha$ \textit{непокрытия} истинного значения $\theta$ сверху или снизу, то есть
\begin{equation}
  \prob{\theta < \what \theta_\tx{lo}} \dot =\, \alpha \quad \tx{ и }  \quad  \prob{\theta > \what \theta_\tx{up}} \dot =\, \alpha
\end{equation}
Можно оценить качество приближенных доверительных интервалов на основании того, насколько они удовлетворяют (14.16). Можно показать, что интервалы $\bca$ имеют второй порядок точности. Это означает, что отклонение от (14.16) сходится к нулю со скоростью $1/n$, (с увеличением размера выборки $n$) то есть 
\begin{equation}
  \prob{\theta < \what \theta_\tx{lo}} \dot =\, \alpha + \frac{c_\tx{lo}}{n} \quad \tx{ и }  \quad  \prob{\theta > \what \theta_\tx{up}} \dot =\, \alpha + \frac{c_\tx{up}}{n}
\end{equation}
для двух констант $c_\tx{lo}$ и $c_\tx{up}$. Стандартный и процентильный методы имеют лишь \textit{первый порядок точности}, поэтому ошибки оказываются на порядок выше:
\begin{equation}
    \prob{\theta < \what \theta_\tx{lo}} \dot =\, \alpha + \frac{c_\tx{lo}}{\sqrt n} \quad \tx{ и }  \quad  \prob{\theta > \what \theta_\tx{up}} \dot =\, \alpha + \frac{c_\tx{up}}{\sqrt n},
\end{equation}
где константы $c_\tx{lo}$ и $c_\tx{up}$ могут отличаться от тех, которые были ранее. Разница между первым и вторым порядком точности имеет не только теоретический характер. Она также приводит к улучшенной аппроксимации точных границ тогда, когда они существуют, как в таблице 14.2.

Метод бутстреп-t имеет второй порядок точности, однако не обладает свойством сохранения отображения. Процентильный метод обладает, однако не имеет второй порядок точности; как и стандартный метод. $\bca$ метод обладает обоими преимуществами. На текущий момент метод $\bca$ рекомендуется к универсальному использованию, в особенности для непараметрических задач. Нельзя сказать, что метод идеален или не может быть модифицирован: в разделе 25.6 главы 25 используется дополнительное применение бутстрепа для улучшения результатов, полученных  с помощью $\bca$ и $\abc$ методов. % В задаче 14.13 о трудностях, которые могут возникнуть с $\bca$ интервалами в экстремальных ситуациях.

Стандартный вызов функции \texttt{bcanon} имеет вид
\begin{equation}
  \texttt{bcanon}(\texttt{x}, \texttt{nboot}, \texttt{theta}),
\end{equation}
где \texttt{x} --- данные, \texttt{nboot} --- число бутстреп репликаций, \texttt{theta} --- вид статистики $\what \theta$. Больше подробностей --- в приложении.


 

