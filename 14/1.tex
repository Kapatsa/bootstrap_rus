\section{Введение}
Одной из основных целей теории бутстрепа является автоматическое создание хороших доверительных интервалов. <<Хорошо>> означает, что бутстреп интервалы должны быть близки к точным доверительным интервалам в тех особых ситуациях, когда статистическая теория дает точный ответ, и должны иметь надежные вероятности покрытия в любых ситуациях. Ни метод бутстреп-t главы 12, ни метод процентилей главы 13 не соответствуют этим критериям. Бутстреп-t интервалы имеют хорошие теоретические вероятности покрытия, но на практике имеют тенденцию быть неустойчивыми. Процентильные интервалы более устойчивы, но имеют менее удовлетворительные свойства покрытия.

В этой главе обсуждается улучшенная версия процентильного метода, называемого $\bca$ (аббревиатура, bias–corrected and accelerated). Интервалы $\bca$ являются существенным улучшением по сравнению с процентильными интервалами как в теоретическом плане, так и на практике. Они близки к приведенным выше критериям качества, хотя точность их покрытия все еще может быть неустойчивой для небольших размеров выборки. (Возможны улучшения, как показано в главе 25.) Простой компьютерный алгоритм под названием \texttt{bcanon} производит интервалы $\bca$, затрачивая для этого немного больше усилий, чем для процентильных интервалов. Мы также обсудим метод под названием $\abc$ (аббревиатура, approximate bootstrap confidence intervals) который значительно уменьшает объем вычислений, необходимых для интервалов $\bca$. Глава заканчивается применением этих методов к реальной задаче.