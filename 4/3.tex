\section{Оценка стандартной ошибки среднего}

Предположим, что у нас есть случайная выборка чисел $F \rightarrow x_1, x_2, \cdots, x_n$, например контрольные измерения $n = 9$ для данных о мышах из таблицы 2.1. Мы вычисляем оценку $\bar x$ для математического ожидания $\mu_F$, равного 56.22 для данных о мышах, и хотим знать стандартную ошибку $\bar x$. Формула (5.4) $se_F (\bar x) = \sigma_F/\sqrt{n}$ включает неизвестное распределение $F$ и поэтому не может использоваться напрямую. 

На этом этапе мы можем использовать принцип плагина: мы подставляем $\hat F$ вместо $F$ в формуле $se_F (\bar x) = \sigma_F/\sqrt{n}$. Плагин оценка $\sigma_F = [E_F (x- \mu_F)^2]^{1/2}$ равна
\begin{equation}
    \hat\sigma=\sigma_{\hat F}=\{\frac{1}{n}\sum_{i=1}^n(x_i-\bar x)^2\}^{1/2},
\end{equation}
поскольку $\mu_{\hat F}= \bar x$ и $E_{\hat F}g (x) = \frac{1}{n}\sum_{i=1}^n g (x_i)$ для любой функции $g$. Это дает оценку стандартной ошибки $\widehat{se} (\bar x) = se_{\hat F} (\bar x)$, 
\begin{equation}
    \widehat{se} (\bar x) = \sigma_{\hat F}/\sqrt{n}=\{\sum_{i=1}^n(x_i-\bar x)^2/n^2\}^{1/2}.
\end{equation}
Для контрольной группы данных о мышах $\widehat{se} (\bar x) = 13.33$.

Формула (5.12) немного отличается от обычной оценки стандартной ошибки (2.2). Это потому, что $\sigma_F$ обычно оценивается как $\bar\sigma=\{\sum(x_i-\bar x)^2/(n-1)\}^{1/2}$, а не как $\hat\sigma$, (5.11). Деление на $n - 1$ вместо $n$ делает $\bar\sigma^2$ несмещенной для $\sigma_F^2$. Для большинства целей $\hat\sigma$ так же хороша, как $\bar\sigma$ для оценки $\sigma_F$. 

Обратите внимание, что мы использовали принцип плагина дважды: сначала для оценки математического ожидания $\mu_F$ с помощью $\mu_{\hat F} = \bar x$, а затем для оценки стандартной ошибки $se_F (\bar x)$ с помощью $se_{\hat F} (\bar x)$. Бутстреп оценка стандартной ошибки, о которой идет речь в главе 6, сводится к использованию принципа плагина для оценки стандартной ошибки произвольной статистики $\hat\theta$. Здесь мы видели, что если $\hat\theta = \bar x$, то этот подход приводит к (почти) обычной оценке стандартной ошибки. Как мы увидим, преимущество бутстрепа в том, что его можно применить практически к любой статистике $\hat\theta$, а не только к среднему значению $\bar x$. 