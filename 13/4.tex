\section{Обратный процентильный интервал}
Процентильный доверительный интервал для $\theta$ имеет $\widehat{G}^{-1}(\alpha)$ в качестве левой и $\widehat{G}^{-1}(1 - \alpha)$ в качестве правой граничной точки. Бутстреп-t метод из предыдущей главы использует бутстреп для оценки распределения стьюдентизированной (приблизительной) опорной точки, а затем инвертирует опорную точку для получения доверительного интервала. Чтобы сравнить эти два интервала, подумайте, что произойдет, если мы упростим  бутстреп-t и построим его на основе $\widehat{\theta} - \theta$. То есть положим знаменатель опорной точки равным 1. Легко показать, что результирующий интервал есть:
\begin{gather}\label{13.9}
[2\widehat{\theta} - \widehat{G}^{-1}(1 - \alpha),\ 2\widehat{\theta} - \widehat{G}^{-1}(\alpha)].
\end{gather}
Обратите внимание, что если у $\widehat{G}$ длинный хвост \textit{вправо}, то у этого интервал длинный хвост \textit{влево}, такое поведение противоположно поведению процентильного интервала. 

Какой из них использовать? В целом не один из этих интервалов не работает хорошо: в последнем случае мы должны использовать $(\widehat{\theta} - \theta) /  \widehat{\text{se}}$, а не $\widehat{\theta} - \theta$ (см. раздел 22.3), а процентильный интервал может потребовать дополнительных уточнений, как описано в следующей главе. Однако на некоторых простых примерах мы видим, что процентильный интервал предпочтительнее. Для коэффициента корреляции, обсуждаемого в главе 12 (для нормальной модели) величина $\widehat{\varphi} - \varphi$, где $\varphi$ преобразование Фишера (\ref{12.24}), хорошо аппроксимируется нормальным распределением и, следовательно, интервал процентилей является точным. Напротив, величина $\widehat{\theta} - \theta$ далека от основной, так что интервал (\ref{13.9}) не очень точный. Другой пример касается вывода для медианы. Процентильный интервал близок к интервалу, основанному на статистике, в то время как (\ref{13.9}) --- наоборот. Подробноее у Эфрона (1979).