\section{Свойство инвариантности относительно преобразования}
Вернемся еще раз к правой части рисунка 13.2. $95\%$ процентильный интервал для $\varphi$ оказывается равным $[-0.29, 0.73]$. Что бы мы получили, если бы преобразовали это обратно в масштаб $\theta$ с помощью обратного преобразования $\exp(\varphi)$? Преобразованный интервал равен $[-0.75, 2.07]$, что в точности соответствует интервалу процентилей для $\theta$. Другими словами, процентильный интервал \textit{инвариантен относительно преобразования}: интервал процентилей для любого (монотонного) преобразования параметров $\varphi = m(\theta)$ --- это просто процентильный интервал для $\theta$, отображаемый $m(\theta)$:
\begin{gather}
    [\what{\varphi}_{\%,\ \text{lo}}, \what{\varphi}_{\%,\ \text{up}}] = 
    [m(\what{\theta}_{\%,\ \text{lo}}), m(\what{\theta}_{\%,\ \text{up}})].  
\end{gather}
То же свойство сохраняется и для эмпирических процентилей, основанных на B бутстреп выборках.

Как мы видели в приведенном выше примере для коэффициента корреляции, 
стандартный нормальный интервал не учитывает преобразование. 
Это свойство является важным практическим преимуществом метода процентилей.