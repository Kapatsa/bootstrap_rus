
\section{Библиографические примечания}
Справочные сведения о бутстреп доверительных интервалах приведены в библиографических примечаниях в конце главы 22.

%\section{Задачи}
%13.1 Докажите, что перцентильный интервал (\ref{13.10}) сохраняет диапазон. Используйте это, чтобы проверить лемму о процентильном интервале.

%13.2 (a) Предположим, мы находимся в одновыборочной непараметрической ситуации из главы 6, где $F \in \mathbf{x}, \widehat{\theta} = t(\widehat{F})$. Почему соотношение (\ref{13.8}) может не выполняться именно в этом случае?
%(b) Приведите пример параметрической ситуации $\widehat{P} \in \mathbf{x}$, в которой (\ref{13.8}) точно выполняется.

%13.3 Докажите, что примерный процентильный интервал (\ref{13.4})  сохраняет диапазон, как определено в (13.10).

%13.4 Проведите исследование, подобное тому что приведено в Таблице 13.3, для следующей задачи: $x_{1}, x_{2},\ldots,x_{20}$ независимы с экспоненциальным распределением, имеющим среднее значение $\theta$. (Экспоненциальная переменная со средним значением $\theta$ может быть определена как $-\theta \log U$, где $U$ равномерна распределена на [0, 1]. Интересующий параметр $\theta = 1$. Вычислить стандартный и процентильный интервал, и объяснить свои результаты.

%13.5 Предположим, что мы оцениваем распределение $\widehat{\theta} - \theta$ по бутстреп распределению $\hat{\theta}^{*} - \hat{\theta}$. Обозначим $\alpha$ - процентиль $\hat{\theta}^{*} - \hat{\theta}$ как $\hat{H}^{-1}(\alpha)$.f Покажите, что интервал для $\theta$, полученный в результате обращения отношения:
%\begin{gather}\label{13.12}
%\hat{H}^{-1}(\alpha)\le \hat{\theta} - \theta \le \hat{H}^{-1}(1 - \alpha),
%\end{gather} дается выражением (\ref{13.9}).

