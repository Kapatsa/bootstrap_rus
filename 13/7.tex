\section{Свойство сохранения диапазона}
Для некоторых параметров существует ограничение на значения, которые 
параметр может принимать. Например, значения коэффициента корреляции 
лежат в интервале [-1, 1]. Ясно, что было бы желательно, чтобы метод 
находил такие доверительные интервалы, которые попадают в допустимый диапазон: 
такой интервал называется \textit{сохраняющим диапазон} интервал. 
Процентильный интервал сохраняет диапазон, поскольку а) оценка с помощью 
подстановки $\widehat{\theta}$ имеет то же ограничение на диапазон, что и 
$\theta$ б) его граничные точки являются значениями бутстреп статистики 
$\widehat{\theta}^{*}$, которые снова подчиняются тому же ограничению диапазона, 
что и $\theta$. Напротив, стандартный интервал не обязательно должен сохранять 
диапазон. Процедуры, сохраняющие диапазон, обычно более точны и надежны.

