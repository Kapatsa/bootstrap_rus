\section{Обсуждение}
Метод процентилей не последнее слово в бутстреп доверительных интервалах. Есть и другие причины, по которым стандартные интервалы могут выйти из строя, помимо ненормальности данных. Например, $\widehat{\theta}$ может быть смещенной нормальной оценкой:
\begin{gather}\label{13.11}
\widehat{\theta} \sim \mathrm{N}(\theta+\text{bias},\  \widehat{\text{se}}^{2}),
\end{gather}
в этом случае никакое преобразование $\varphi = m(\theta)$ не сможет исправить ситуацию. В главе 14 обсуждается расширение метода процентилей, который автоматически обрабатывает как смещение, так и преобразования. Дальнейшее расширение позволяет стандартной ошибке в (\ref{13.11}) изменяться вместе с $\theta$, а не оставаться постоянной. Это последнее расширение будет иметь важное теоретическое преимущество.

