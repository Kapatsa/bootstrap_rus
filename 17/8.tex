\section{Обсуждение}
Оценки ошибки предсказания, которые были описаны в этой главе, являются значительными улучшениями видимой частоты ошибок. Неясно, какой из этих конкурирующих методов лучше. Асимптотически методы одинаковы, но для выборок небольшого размера могут вести себя по разному. Имитационные эксперименты показывают, что кросс-валидация дает примерно несмещенную оценку, но она может иметь большой разброс. Простой бутстреп метод имеет меньшый разброс, но может быть сильно смещен вниз. Усовершенствованный бутстреп подход хоть и дает лучший результат, но все еще смещен вниз. В немногочисленных исследованиях, которые были проведены на сегодняшний день, оценка 0.632 показала лучший результат среди всех методов, но нам нужно больше доказательств, прежде чем давать какие-либо рекомендации.

