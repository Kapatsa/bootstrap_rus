\section{Введение}
До сих пор наше обсуждение было сосредоточено на нескольких статистических показателях точности: стандартная ошибка, смещение и доверительный интервал. Это меры точности параметров модели. Ошибка предсказания --- это величина, которая показывает, насколько хорошо модель предсказывает значение отклика для будущего наблюдения. Она часто используется для выбора модели, поскольку разумно выбрать модель с наименьшей ошибкой предсказания.

Кросс-валидация --- это стандартный инструмент для оценки ошибки предсказания. В связи с увеличением доступной вычислительной мощности эта идея (предшествующая бутстрепу) стала набирать популярность в последние годы. В этой главе мы обсудим, кросс-валидацию, бутстреп и некоторые другие методы, которые оценивают ошибки предсказания.

В регрессионных моделях ошибка предсказания --- это математическое ожидание квадрата разности истинного значения и предсказания, которое было получено с помощью модели: 
\begin{equation}
\tx{PE} = \tx{E}(y - \what{y})^{2}.
\end{equation}
Математическое ожидание относится к повторной выборке из генеральной совокупности. Ошибка предсказания также возникает в задаче классификации, когда ответом является один из классов $k$. Например, в политическом опросе возможными ответами могут быть республиканец, демократ или независимый кандидат. Обычно, в задачах классификации ошибка определяется как вероятность неправильной классификации
\begin{equation}
\tx{PE} = \tx{Prob}(\what{y} \neq y),
\end{equation}
она также называется долей неправильно классифицированных наблюдений. Методы, описанные в этой главе, применимы как для определения ошибок предсказания, так и для других ошибок. Мы начнем с интуитивного описания методов, а затем рассмотрим их более подробно в разделе 17.6.2.
